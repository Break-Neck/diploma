\documentclass[pdftex,ptm,14pt,a4paper]{extreport}

\usepackage[
bookmarks=true, colorlinks=true, unicode=true,
urlcolor=black,linkcolor=black, anchorcolor=black,
citecolor=black, menucolor=black, filecolor=black,
]{hyperref}
\usepackage{graphicx}
\usepackage{enumerate}
\usepackage{cmap}
\usepackage{cite}
\usepackage[utf8]{inputenc}
\usepackage[english,russian]{babel}
    \addto{\captionsenglish}{\renewcommand{\bibname}{Литература}}
    \addto\captionsenglish{\renewcommand{\figurename}{Рис.}}
    \addto\captionsenglish{\renewcommand{\contentsname}{Содержание}}
    \addto\captionsenglish{\renewcommand{\proofname}{Доказательство}}
\usepackage{geometry}
	\geometry{left=2.5cm}
	\geometry{right=1cm}
	\geometry{top=2cm}
	\geometry{bottom=2cm}
\usepackage{setspace}
    \setstretch{1.5}
\usepackage{amsthm}
\usepackage{amssymb}
\usepackage{amsmath}
\usepackage{hyperref}

\DeclareMathOperator*{\argmin}{argmin}
\setlength{\parindent}{1.25cm}

\begin{document}

\begin{titlepage}

\newpage

\begin{center}
МИНИСТЕРСТВО ОБРАЗОВАНИЯ И НАУКИ РОССИЙСКОЙ ФЕДЕРАЦИИ \\
\vspace{0.5cm}
ГОСУДАРСТВЕННОЕ ОБРАЗОВАТЕЛЬНОЕ УЧРЕЖДЕНИЕ \\*
ВЫСШЕГО ПРОФЕССИОНАЛЬНОГО ОБРАЗОВАНИЯ\\*
"МОСКОВСКИЙ ФИЗИКО-ТЕХНИЧЕСКИЙ ИНСТИТУТ \\*
(ГОСУДАРСТВЕННЫЙ УНИВЕРСИТЕТ)" \\*
\vspace{0.5cm}
ФАКУЛЬТЕТ ИННОВАЦИЙ И ВЫСОКИХ ТЕХНОЛОГИЙ \\*
КАФЕДРА АНАЛИЗА ДАННЫХ \\*
\hrulefill
\end{center}


\vspace{4em}

\begin{center}
\Large Выпускная квалификационная работа по направлению 01.04.05 <<Прикладные математика и информатика>> \linebreak НА ТЕМУ:
\end{center}

\vspace{2.5em}

\begin{center}
\textsc{\large{\textbf{Предсказание финансовых показателей по новостям}}}
\end{center}

%\vspace{5em}

\begin{flushleft}
Студент \hrulefill Жигунов А.Л. \\
\vspace{1.5em}
Научный руководитель \hrulefill Трофимов И.Е.\\
\end{flushleft}

\vspace{\fill}

\begin{center}
МОСКВА, 2018
\end{center}

\end{titlepage}

\section{Аннотация}

С наблюдаемым за последние десятилетия ростом количества оперативно публично доступной информации появились возможности к предсказанию влекомых ей изменений и автоматическому принятию решений на ее основе.
Данная работа посвящена предсказанию одного из объективных финансовых показателей -- относительных курсов валют, который был выбран по двум причинам: это легко доступная информация, на которую оказывают очевидное влияние мировые новости, а также, при удачном предсказании курса, можно принимать на их основе решения по торговле в автоматическом режиме.

Идея извлечения информации из новостей и ее приложение к анализу финансовых рынков не нова, она
в разное время рассматривалась и экономистами, и специалистами по статистике, а позже и машинному обучению.
Итогом со стороны теоритических экономистов стали различные концепции,
описывающие проблемы данных предсказаний при различной структуре рынков; в то же время на практике
были достигнуты определенные успехи.

В данной работе рассматривается задача предсказания изменения официального курса доллара к российскому рублю
на основе новостей за предшествующий период.

\tableofcontents

\chapter{Обзор теории}

\section{Задача бинарной классификации}

В начале напомним постановку задачи бинарной классификации\cite{bin_classif}: пусть задано множество объектов $\mathcal{X}$ и
множество классов $\mathcal{Y} = \{0,1\}$.  Существует \textit{целевая функция} $y^*:\mathcal{X}\to\mathcal{Y}$,
значения которой $y_i, i=1,\ldots,n$, известны на некотором конечном множестве
$X = \{x_1,\ldots,x_n\} \subset \mathcal{X}$ --- \textit{обучающей выборке}. Задача состоит в построении вычислимой функции
$y:\mathcal{X}\to\mathcal{Y}$, приближающей $y^*$ на множестве $\mathcal{X}$.
Стандартно, $x\in\mathcal{X}$ представляется как вектор из $m$ значений $x=(x^1,\ldots,x^m)$, тогда обучающая выборка
представима матрицей $X\in \mathbb{R}^{n\times m}$.

\section{Оценка качества}

Для оценки качества стандартной техникой является разбиение выборки на 2 непересекающиеся части
$X = X_\text{train+val} \sqcup X_\text{test}$, обучение и подбор гиперпараметров проводится на части $X_\text{train+val}$,
а качество выбранного алгоритма оценивается на $X_\text{test}$.
Подбор гиперпараметров также проводится на отдельной от $X_\text{test}$ части выборки, чтобы обеспечить $X_\text{test}$ эффект
данных, которые алгоритм на обучении никак не использовал:
$X_\text{train+val} = X_\text{train} \sqcup X_\text{valid}$, где по результатам
запусков на части $X_\text{valid}$ и выбираются лучшие гиперпараметры. Классическим улучшением является усреднение результатов
по нескольким сбалансированным разбиениям.

В задачах, где у данных есть временная специфика и требуется избежать ``подглядывания'' в будущее упомянутая выше техника,
называемая кроссвалидацией, приобретает дополнительное свойство линейного разбиения по времени: $\forall x_\text{train}\in X_\text{train}, x_\text{valid}\in X_\text{valid}, x_\text{test}\in X_\text{test}:
t(x_\text{train} ) < t(x_\text{valid} ) < t(x_\text{test} )$, где $t(x)$ -- время для объекта $x$.
\textbf{В дальнейшем для удобства будем предполагать, что индексация элементов $X$ согласована со временем:
$i<j\Rightarrow t(x_i)<t(x_j)$.}

Для данной задаче будет использоваться следующая модификация указанной выше общей техники: проводится
$|X_\text{valid}|$ раундов, на раунде с номером $k$ исследуемый классификатор обучается на множестве
\[X_\text{train} \sqcup \{x_i \in X_\text{valid} \mid |X_\text{train}| < i < |X_\text{train}| + k\} =
\{x_i\in X \mid i < |X_\text{train}| + k\} \]
и используется для предсказания на объекте $x_{|X_\text{train}|+k}$. Данный процесс связан с тем, как подобная система
может применяться на практике: для предсказания для какого-то дня она обучается на всей его известной истории, после чего
выдает для него предсказание. Следует отметить, что при этом $X_\text{valid}$ выборка также будет задействована при обучении.

Для оценки качества будут использоваться 2 величины: $q_\text{acc}$ и $q_\text{macd}$. $q_\text{acc}$ равно доле объектов
множества, на которых алгоритм дал правильный ответ. Расчет же $q_\text{macd}$ опирается на сравнение торговых стратегий,
поэтому, будем считать, что у нас имеются $p_i$ --- курсы какого-либо товара, а упомянутые
$y_i=\mathbb{I}_{\{p_i > p_{i-1}\}}$.
Тогда можно говорить о проведении торговых операций, и, соответственно, о торговых стратегия, построенных на предсказанных $y_i$.

\subsection{MACD}

Индикатор MACD\cite{macd} основан на скользящих средних и показывает изменение тренда в поведении цен. Пусть дан $p_i$ ---
ряд цен за какой-то период и $ev_k(a_i)$ -- экспоненциальное среднее части числового ряда $a_{i-(k-1)},\ldots,a_i$. Тогда
\begin{equation}
\begin{split}
M_i &= ev_s(p_i) - ev_l(p_i), s < l \\
S_i &= ev_a(M_i)
\end{split}
\end{equation}

Если для какого-то $i$ изменяется знак разности $M_i - S_i$, то это трактуется как фиксация смены тренда для $p_i$:
\begin{itemize}
\item если знак сменился на положительный, то тренд считается сменившимся на возрастание
\item если знак сменился на отрицательный, то тренд считается сменившимся на понижение
\end{itemize}

Стандартными для подневного анализа считаются значения $s=12, l=26, a=9$.

\begin{figure}
\centering
\includegraphics[width=0.6\linewidth]{macd}
\caption{Иллюстрация MACD с параметрами $l=26, s=12, a=9$\label{fig:macd}}
\end{figure}

На рисунке \ref{fig:macd} изображены примеры колебаний данных, их усреднения с периодами 12 и 26 (сверху) и разница этих средних,
ее усреднение с периодом 9 и их разность (снизу).


\subsection{Методика проведения сравнения стратегий}

Производится сравнение 2 стратегий. Обеим дается в начале периода одинаковая сумма денег, после чего:
\begin{itemize}
\item стратегия MACD реагирует на соответствующие сигналы и принимает решение, купить или продать предмет оборота
\item стратегия основанная на тестируемом классификаторе действует жадно, опираясь на его предсказание завтрашнего
поведения курса $y_{i+1}$
\end{itemize}
Обе стратегии ведут себя наивно-жадно, в каждой операции задействуя все имеющиеся средства для обмене.

Далее для каждого дня берется отношение количества денежных средств, имеющихся у каждой из стратегий (если средства были
вложены в покупку, то они переводятся в денежный эквивалент по курсу этого дня). Для определения сравнительного состояния
стратегий на определенный день, рассматривается отношение их денежных средств; это удобный показатель, не зависящий
от текущего курса, а только от ``объема благ'', накопленных каждой стратегией.

Итак, если $a^{\text{ML}}_i, b^{\text{MACD}}_i$ --- денежные средства стратегий на день $i$, тогда для периода из $n$ дней
\begin{equation}
q_\text{macd} = \frac{\sum_i^n \frac{a^{\text{ML}}_i}{b^{\text{MACD}}_i}}{n}
\end{equation}

Таким образом, данную величину можно понимать как средний выигрыш по сравнению с торговлей по MACD.

\section{Линейные модели для предсказания}

Одними из самых простых, но в тоже время достаточно мощных моделей в машинном обучении являются линейные модели\cite{linear_cls}.
Для представления $x\in\mathcal{X}=(x^1,\ldots,x^m)$ обученный классификатора характеризуется вектором весов $w\in\mathbb{R}^m$,
и его предсказание получается как ($f(a) : \mathbb{R}\to\mathcal{Y}$ -- функция,
характеризующая подсемейство классификаторов)
\begin{equation}
y=f(w^Tx)
\end{equation}

Один из методов обучения данных классификаторов связан с представлением искомых $w^*$ как решения задачи минимизации
по обучающей выборке:
\begin{equation}
w^*=\argmin_w \sum_{i=1}^{|X|} L(y_i,w^Tx_i) + R(w)
\end{equation}

Выше
\begin{itemize}
\item $L(y_i, y)$ -- \textit{функция потерь}, оценивающая ошибки предсказания.
Важно, что она применяется к соответствующему $f(a)$. Примеры:
\begin{itemize}
\item log loss: $L(y_i, y) = -y_i\log{y}-(1-y_i)\log{(1-y)}$, $y$ имеет смысл предсказанной вероятности принадлежности
выделенному классу
\item hinge loss: $L(y_i, y) = \max(0, 1 - yy_i)$, здесь для удобства $y \in \{-1,1\}$.
\end{itemize}
\item $R(w)$ -- регуляризация\cite{regularization}, часть, предотвращающая деградацию качества на данных, на которых
алгоритм не обучался. Примеры ($\lambda_1,\lambda_2\geq0$):
\begin{itemize}
\item $L_2: R(w)=\lambda_2\|w\|_2$
\item $L_1: R(w)=\lambda_1\|w\|_1$
\item $L_1 + L_2: R(w)=\lambda_1\|w\|_1+\lambda_2\|w\|_2$
\end{itemize}
Последние 2 функции могут приводить к занулению некоторой части координат вектора $w^*$.
\end{itemize}

Наиболее распространенным в последнее время стало решение данной задачи с помощью SGD\cite{sgd}. 

\section{Векторизация текста}

Важным этапом для работы с любыми текстовыми источниками является процесс их векторизации --- представления их в векторном
виде. Будем считать, что нам дан текст на английском языке. Тогда одним из общепринятых методов перевода будет следующая
последовательность преобразований:
\begin{enumerate}
\item Токенизация\cite{info_retrieval}: для текста ``\texttt{About seven bits of important information is contained here!}'' получается
его разбиение на части, называемые токенами: (``\texttt{about}'', ``\texttt{seven}'', ``\texttt{bits}'', ``\texttt{of}'',
``\texttt{important}'', ``\texttt{information}'', ``\texttt{is}'', ``\texttt{contained}'', ``\texttt{here}'). Различные детали, как считать ли знаки препинания отдельными токенами, или переводить ли токены в 1 регистр, зависят от
токенизатора.
\item Лемматизация\cite{info_retrieval}: текстовые токены приводятся в ``нормальную форму'':
(``\texttt{about}'', ``\texttt{seven}'', ``\texttt{bit}'', ``\texttt{of}'',
``\texttt{important}'', ``\texttt{information}'', ``\texttt{be}'', ``\texttt{contain}'', ``\texttt{here}').
Это довольно нетривиальный процесс, не всегда существующие машинные техники получают результат, корректный с точки
зрения человека. Альтернативой является стемминг, при котором различные формы слова приводятся к одинаковому виду,
который необязательно является корректным словом.
\item Фильтрация стопслов: выбрасываются слова, не несущие самостоятельного смысла, например, артикли и местоимения:
(``\texttt{seven}'', ``\texttt{bit}'', ``\texttt{important}'', ``\texttt{information}'', ``\texttt{contain}'')
\item Дополнительная фильтрация слов: зачастую, происходит фильтрация по частоте встречаемости, для избавления от слишком редких
слов
\item Превращение текста в вектор: пусть для корпуса из $n$ текстов каждый прошел описанную выше обработку $w_i \in W,
w_i=(w_i^1,\ldots,w_i^{l_i})$ --- получившиеся последовательности, тогда классическое преобразование
полученной последовательности элементов $w_i$ в вектора Bag-of-Words\cite{info_retrieval}:
\begin{equation}
v\in\mathbb{R}^m: m=\left|\bigcup_{i,j} \{w^j_i\}\right|,\ v_i - \text{количество вхождения терма } i\text{ в } w_i
\end{equation}
\end{enumerate}

\section{Понижение размерности}

Зачастую при работе с большим количеством признаков, например, при работе с текстами, появляется желание уменьшить
размер признакового пространства без большой регрессии качества на данных. Одним из классических методов является
понижение размерности с помощью SVD\cite{svd}.

В нем признаковая матрица $X\in\mathbb{R}^{n\times m}$ представляется в виде $X=U\Sigma V^*$, где $\Sigma$ --- диагональная
матрица, где на диагонали расположены сингулярные значения матрицы упорядоченные по убыванию, а $U$ и $V$ --- матрицы 
соответствующих им левых и правых сингулярных векторов. Тогда можно показать, что для $X_m=U_m\Sigma_m V_m$,
где матрицы с индексов $m$ получены как подматрицы из элементов с индексами $i,j=\overline{1m}$,
$X_m$ является наилучшим приближением в смысле Фробеунисовой нормы ($\|A\|_F = \sqrt{\sum_{i,j} a^2_{ij}}$)
матрицы $X$ среди матриц ранга $m$.

Одним из преимуществ использования SVD является оценка потерь при понижения размерности, т.к. $\|Ax\|_2 \leq \|A\|_F\|x\|_2$,
и по доле суммы квадратов оставшихся сингулярных значений можно оценить величину потери для $Ax$, что важно для линейной
модели.

\iffalse

\chapter{Обзор предшествующих работ в области}

\section{Гипотеза эффективного рынка}

Гипотеза эффективного рынка (Efficient-market hypothesis) --- экономическая гипотеза, выдвинутая Юджином Фама \cite{EFH}, согласно которой существенная информация немедленно и в полной мере отражается на рыночной курсовой стоимости ценных бумаг. Различают 3 разных формы гипотезы:

\begin{enumerate}
\item Сильная форма -- \textit{вся} (в том числе, не общедоступная!) существующая информация отражается в цене актива,
\item Средняя форма -- текущая цена отражает только прошлую и текущую \textit{общедоступную} информацию,
\item Слабая форма -- цена в полной мере отражает только информацию из прошлого.
\end{enumerate}

Так как в данной работе рассматривается предсказание будущих значений показателей на основе только публичной информации,
то при выполнении сильной гипотезы на рынке для построенных стратегий невозможен выигрыш против игроков, использующих в том
числе и закрытую информацию.

\section{Обзор близких работ}

Как пример известных результатов по схожей тематике в пример можно привести:

\paragraph{Предсказание рисков \cite{risk_predict}}

В данной работе предсказывается стандартное отклонение курса акций за некоторый период по новостям за этот период.
Несмотря на использование довольно простых моделей (для интерпретируемости) на основе только текстовой информации получается
довольно хорошее качество предсказания. Показано, что при добавлении к модели, основанной только на численных метриках,
 признаков, полученных из обработки новостей, качество предсказания увеличивается.

Отличие от текущей работы --- предсказывается стандартное отклонение, а не само изменение значения.

\paragraph{Выделение новостей \cite{select_important_news}}

Авторы учатся выделять новости, коррелирующие с изменениями финансовых показателей, достигая достаточных для
применения на практике результатов.

Отличие от текущей работы --- производится выделение новостей коррелирующих с изменением курса без учета характера
самого изменения.

\paragraph{Предсказание колебаний по Twitter \cite{stock_from_twitter}}

На основе готовой разметки сообщений Twitter (на 4 категории по эмоциональной окраске) предсказываются колебания курса --- разность курса и установленного для них тренда.

Отличие от текущей работы --- используются не новости, а сообщения Twitter; предсказываются внетрендовые колебания,
а не полные изменения.

\fi

\chapter{Вычислительный эксперимент}

В качестве языков программирования в работе использовались C++ и Python\cite{python}.

\section{Данные}

\subsection{Данные новостей}

В качестве новостных данных были взяты новости из архива Reuters на английском языке за 2010-2017 годы\cite{reuters}.
Так как структура и стилистика текста в данном случае были довольно стандартными для общей публицистики,
то в обработке практически не потребовалось каких-то нестандартных для NLP приемов.

Датасет был получен с помощью робота, написанного с использованием библиотеки Scrapy\cite{scrapy}.

\subsection{Данные по торговому курсу}

В качестве цели предсказания были взяты изменения значения официального курса доллара к российскому рублю, доступные
на сайте Центрального Банка РФ\cite{courses}.
Этот курс публикуется в 11:30 по московскому времени и действует на следующие календарные дни, до вступления в силу следующего
официального курса.

Так как этот курс установлен не для всех дней (выпадают выходные, праздники и прочее),
то для определения дня, на курс для которого мы считаем повлиявшей новость, предлагается делать следующее:
новость \textit{относится} к тому дню, в который для нее происходит ``следующее'' 11:30. При этом, так как для практической
применимости результатов требуется предсказывать с запаздыванием, позволяющим провести торговые операции,
то целью предсказания мы будем считать изменение курса в день для которого известен установленный курс, и
следующий за днем, к которому относится новость.

Другими словами, если $T_1,\ldots , T_n$ -- дни, для которых известен курс, то:
\begin{itemize}
\item новость, опубликованная в день $T_i$ ранее 11:30, будет участвует в предсказании для дня $T_{i+1}$
\item новость, опубликованная в день $T_i$ позднее 11:30, будет участвует в предсказании для дня $T_{i+2}$
\end{itemize}

Значение, которое требуется предсказать -- это знак изменения курса, таким образом задача сводится к бинарной классификации.

\subsection{Разбиение данных}

Разбиение на $X_\text{train}, X_\text{valid}, X_\text{test}$ было произведено в соотношении $2:2:1$.
Такие нетипично большие размеры валидационной
и тестовых частей были вызваны необъективностью показателя $q_\text{macd}$ на малых отрезках, из-за чего и потребовалось выделить для каждой из них значительную часть выборки.

\section{Обработка текстовых данных}

\subsection{Векторизация новостей}

Этапы векторизации текста новостей, для первых 2 этапов была использована их реализация в библиотеке NLTK\cite{nltk}:

\begin{enumerate}

\item Токенизация. Был использован известный метод Treebank\cite{treebank}) с последующим отсеиванием токенов,
содержащих символы кроме буквено-цифровых.
\item Лемматизация. Использовался WordNet\cite{wordnet}, из-за специфики статей результаты были вполне удовлетворительными,
и было решено использовать их, а не результаты стемминга.
\item Учет \texttt{not} и стоп-слова. Было решено произвести фильтрацию стоп-слов, однако до этого последовательности
термов (\texttt{not}, $w_1$, $w_2$) были заменены на ($\text{not\_} w_1, \text{not\_} w_2$) для учета смыслового отрицания
для прилагательных.
\item Добавление биграмм\footnote{пара термов, идущих подряд в тексте} как самостоятельных термов.
\item Частотная фильтрация --- оставить термы со встречаемостью не менее 10000 на весь корпус. В дальнейшем было проверено,
что множество оставшихся термов не уменьшается при уменьшении выборки на 25\%, что позволяет сказать об устойчивости
результата для данного формата данных. В среднем на каждый день приходится 1200 статей.

\end{enumerate}

В результате осталось около 7000 термов, после чего каждый документ $d_i$ был представлен в виде вектора:
\begin{equation}
v_i = v(d_i) = (v_{ij})_{j=1}^{|W|}
\end{equation}
Выше $v_{ij}$ --- количество вхождений терма с номером $j$ в документ $d_i$, что соответствует модели Bag-of-Words.

\subsection{Сведение от новостей к дням}

Так как предсказание делается для дней, то необходимо получить векторное представление дня.
В данной работе был использован один из самых очевидных способов, а именно представлять вектор дня как сумму векторов соответствующих этому дню новостей, деленную на их количество.
\begin{equation}
x_i = \frac{1}{\left|\{v_j | time(v_j)=T_i\}\right|} \sum_{v_j: t(v_j)=T_i} v_j
\end{equation}

\section{Преобразование признаков}

Так как число дней с известным курсом за рассматриваемый период не превосходит 2000, а признаков -- около 7000, то возникла
необходимость в понижении размерности. После понижения размерности с помощью SVD\cite{svd} с сохранением
99\% информации количество оставшихся фич составило 1700, что позволило решать задачу с использованием стандартных
приемов регуляризации. С получившейся матрицей было произведено попризнаковое
масштабирование: для каждого значения признака из него было вычтено его среднее значение по выборке и результат поделен на
выборочное отклонение.

Итоговый линейный классификатор был обучен с помощью SGD\cite{sgd}, параметры регуряризации ($\lambda_1,\lambda_2$ для L1+L2)
и целевая функция (log-loss, hinge-loss) выбирались на кроссвалидации.

\section{Результаты}

На рисунке \ref{fig:vs_macd_on_test} приведен график для 3 классификаторов, численное сравнение в таблице \ref{test_table_compare}:

\begin{itemize}
\item \texttt{Top accuracy} -- лучший по точности предсказания на кроссвалидации,
\item \texttt{Top trade} -- лучший по сравнению с MACD на кроссвалидации,
\item \texttt{Combine} -- лучший по среднему рангов по двум указанным критериям на кроссвалидации;
\end{itemize}

\begin{figure}
\includegraphics[width=\linewidth]{vs_macd_on_test}
\caption{Поведение торговых стратегий по сравнению с MACD на тестовой выборке \label{fig:vs_macd_on_test}}
\end{figure}

\begin{table}
\centering 
\begin{tabular}{| c | c | c |}
\hline
 & Среднее & Отклонение \\ \hline
Top trade & 1.04 & 0.030 \\ \hline
Top accuracy & 1.13 & 0.029 \\ \hline
Combine & 1.13 & 0.031 \\ \hline
\end{tabular}
\caption{Характеристики отношения стратегий к MACD на тестовой выборке\label{test_table_compare}}
\end{table}

Как можно увидеть, лидером является классификатор с максимальной точностью, но смешанная версия не сильно проседает по качеству,
причем на отдельных отрезках опережает остальные.

Также была проверена гипотеза о зависимости предсказания от какого-либо периода (дни недели. начало месяца и проч.).
Стандартный для этого тест Льюнг-Бокса\cite{ljungbox} для индикаторов истинности предсказаний на тестовой выборке
показал $p$-значение 0.88, что позволяет уверенно заключить об отсутствии автокоррекции.

\chapter{Итоги}

Таким образом, в данной работе была продемонстрирована возможность предсказания колебаний курса
валюты --- одного из финансовых показателей, основываясь исключительно на текстово-новостной информации.
Более того, была также произведена демонстрация его применения для создания торговой стратегии,
которая выигрывает в сравнении с классической экономической стратегией MACD.

Несмотря на то, что выигрыш над MACD был достигнут, целью данной работы в большей степени являлась иллюстрация
возможности такого предсказания, а не получение максимальной прибыли над MACD ценой узконаправленных эвристик,
из-за чего в продемонстрированном решении оставлено огромное поле для доработок.
На примере уже существующих работ в данной области для улучшения качества работы можно порекомендовать
смешение текстовых признаков и признаков, описывающих численное поведение курса за небольшую историю;
также нераскрытой остается возможность выделения для новостей признаков, связанных с поведением курса, и изменением
способа представления дня на их основе.

\bibliography{literature}{}
\bibliographystyle{plain}

\end{document}

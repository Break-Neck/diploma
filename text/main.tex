\documentclass[pdftex,ptm,12pt,a4paper]{report}

\usepackage[
bookmarks=true, colorlinks=true, unicode=true,
urlcolor=black,linkcolor=black, anchorcolor=black,
citecolor=black, menucolor=black, filecolor=black,
]{hyperref}
\usepackage{graphicx}
\usepackage{enumerate}
\usepackage{cmap}
\usepackage{cite}
\usepackage[utf8]{inputenc}
\usepackage[english,russian]{babel}
    \addto{\captionsenglish}{\renewcommand{\bibname}{Литература}}
    \addto\captionsenglish{\renewcommand{\figurename}{Рис.}}
    \addto\captionsenglish{\renewcommand{\contentsname}{Содержание}}
    \addto\captionsenglish{\renewcommand{\proofname}{Доказательство}}
\usepackage{amsthm}
\usepackage{amssymb}
\usepackage{amsmath}

\begin{document}

\begin{titlepage}

\newpage

\begin{center}
МИНИСТЕРСТВО ОБРАЗОВАНИЯ И НАУКИ РОССИЙСКОЙ ФЕДЕРАЦИИ \\
\vspace{0.5cm}
ГОСУДАРСТВЕННОЕ ОБРАЗОВАТЕЛЬНОЕ УЧРЕЖДЕНИЕ \\*
ВЫСШЕГО ПРОФЕССИОНАЛЬНОГО ОБРАЗОВАНИЯ\\*
"МОСКОВСКИЙ ФИЗИКО-ТЕХНИЧЕСКИЙ ИНСТИТУТ \\*
(ГОСУДАРСТВЕННЫЙ УНИВЕРСИТЕТ)" \\*
\vspace{0.5cm}
ФАКУЛЬТЕТ ИННОВАЦИЙ И ВЫСОКИХ ТЕХНОЛОГИЙ \\*
КАФЕДРА АНАЛИЗА ДАННЫХ \\*
\hrulefill
\end{center}


\vspace{8em}

\begin{center}
\Large Выпускная квалификационная работа по направлению 01.03.02 <<Прикладные математика и информатика>> \linebreak НА ТЕМУ:
\end{center}

\vspace{2.5em}

\begin{center}
\textsc{\large{\textbf{Предсказание финансовых показателей по новостям}}}
\end{center}

\vspace{7em}

\begin{flushleft}
Студент \hrulefill Жигунов А.Л. \\
\vspace{1.5em}
Научный руководитель д.т.н. \hrulefill Трофимов И.А.\\
\end{flushleft}

\vspace{\fill}

\begin{center}
МОСКВА, 2018
\end{center}

\end{titlepage}

С наблюдаемым за последние десятилетия ростом количества оперативно публично доступной информации появились возможности к предсказанию влекомых ей изменений и автоматическому принятию решений на ее основе.
Данная работа посвящена предсказанию одного из объективных финансовых показателей -- относительных курсов валют, который был выбран по двум причинам: это легко доступная информация, на которую оказывают очевидное влияние мировые новости, а также, при удачном предсказании курса, можно принимать на их основе решения по торговле в автоматическом режиме.

Идея извлечения информации из новостей и ее приложение к анализу финансовых рынков не нова, она
в разное время рассматривалась и экономистами, и специалистами по статистике, а позже и машинному обучению.
Итогом со стороны теоритических экономистов стали различные концепции,
описывающие проблемы данных предсказаний при различной структуре рынков; в то же время на практике
были достигнуты определенные практические успехи.

В данной работе рассматривается задача предсказания изменения официального курса доллара к российскому рублю
на основе новостей за предшествующий период.

\chapter{Обзор предшествующих работ в области}

\iffalse

\section{Гипотеза эффективного рынка}

Гипотеза эффективного рынка (Efficient-market hypothesis) --- экономическая гипотеза, выдвинутая Юджином Фама \cite{EFH}, согласно которой существенная информация немедленно и в полной мере отражается на рыночной курсовой стоимости ценных бумаг. Различают 3 разных формы гипотезы:

\begin{enumerate}
\item Сильная форма -- \textit{вся} (в том числе, не общедоступная!) существующая информация отражается в цене актива,
\item Средняя форма -- текущая цена отражает только прошлую и текущую \textit{общедоступную} информацию,
\item Слабая форма -- цена в полной мере отражает только информацию из прошлого.
\end{enumerate}

Так как в данной работе рассматривается предсказание будущих значений показателей на основе только публичной информации,
то при выполнении сильной гипотезы на рынке для построенных стратегий невозможен выигрыш против игроков, использующих в том
числе и закрытую информацию.

\section{Обзор близких работ}

\fi

Как пример известных результатов по схожей тематике в пример можно привести:

\paragraph{Предсказание рисков \cite{risk_predict}}

В данной работе предсказывается стандартное отклонение курса акций за некоторый период по новостям за этот период.
Несмотря на использование довольно простых моделей (для интерпретируемости) на основе только текстовой информации получается
довольно хорошее качество предсказания. Показано, что при добавлении к модели на основе только численных метрик по статистике
 признаков, полученных из обработки новостей, качество предсказания увеличивается.

Отличие от текущей работы --- предсказывается стандартное отклонение, а не само изменение значения.

\paragraph{Выделение новостей \cite{select_important_news}}

Авторы учатся выделять новости, коррелирующие с изменениями финансовых показателей, достигая достаточных для
применения на практике результатов.

Отличие от текущей работы --- производится выделение новостей коррелирующих с изменением курса без учета характера
самого изменения.

\paragraph{Предсказание колебаний по Twitter \cite{stock_from_twitter}}

На основе готовой разметки сообщений Twitter (на 4 категории по эмоциональной окраске) предсказываются колебания курса --- разность курса и установленного для них тренда.

Отличие от текущей работы --- используются не новости, а сообщения Twitter; предсказываются внетрендовые колебания,
а не полные изменения.

\chapter{Вычислительный эксперимент}

\section{Обработка текстовых данных}

В качестве входных данных были взяты новости из архива Reuters за 2010-2017 годы.

После этого были произведены токенизация (Treebank\cite{treebank}) и лемматизация(WordNet\cite{wordnet}).



\bibliography{literature}{}
\bibliographystyle{plain}

\end{document}